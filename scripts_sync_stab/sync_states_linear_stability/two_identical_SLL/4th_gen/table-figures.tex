\documentclass{article}
 \usepackage{array}
\usepackage[utf8]{inputenc}
\usepackage[english]{babel}
\usepackage[document]{ragged2e}
\usepackage{amsfonts}
\usepackage{geometry}
 \geometry{
 a4paper,
 total={170mm,257mm},
 left=10mm,
 top=20mm,
 }
\usepackage{lscape}
\usepackage{graphicx}
\usepackage{placeins}
\usepackage{float}
\usepackage{subfigure}
\usepackage{stackengine}
\newcommand\xrowht[2][0]{\addstackgap[.5\dimexpr#2\relax]{\vphantom{#1}}}


\begin{document}

%\begin{landscape}

\begin{flushleft}
\begin{tabular}{ |p{1.5cm}||p{4.2cm}|p{5.9cm}|p{3.0cm}|p{3cm}|  }

 \hline
 \multicolumn{5}{|l|}{List of Phase Models' Parameters} \\
 \hline
 Parameter of the phase Model & Value & Parameter of the 4rd Gen Prototype & Value & Comment\\
 \hline\xrowht[()]{15pt}
 $\omega_k$   & $2\pi\cdot 60\cdot10^9$\,radHz    & $f_k=\omega_k/2\pi $ &  $60\cdot10^9 $\,Hz & intrinsic SLL frequency\\
\hline\xrowht[()]{15pt}
 $K_k^{\textrm{\tiny VCO}}$   & $2\pi (\cdot 100... 500)\; \textrm{rad MHz/V}$    & $K_k^{\textrm{\tiny VCO, fine}}$ &  $100... 500\; \textrm{MHz/V}$ & sensitivity of the VCO at $f_{VCO}^{out}$\\
\hline\xrowht[()]{15pt}
$\tau^\textrm{\scriptsize cc, tune}$   & $0\,\dots\,30$\,ns   & $\tau^\textrm{cc, tune}$ &  $0\,\dots\,30$\,ns & cross-coupling time-delay\\
\hline\xrowht[()]{15pt}
$G_k^{\textrm{PD}}$   & $3.24$    & $G_k^{\textrm{PD}}$  & $3.24$ & The gain of the PD\\
\hline\xrowht[()]{15pt}
$G_k^{\textrm{VGA}}$   & $0.5....2$    & $G_k^{\textrm{VGA}}$ & $0.5....2$ & The variable gain of the VGA \\
\hline\xrowht[()]{15pt}
$G_k^{\textrm{LF}}$   & $1$    & $G_k^{\textrm{LF}}$ &  $0\;dB$ & loop filter gain\\
\hline\xrowht[()]{15pt}
$\omega^c$   & $100\,\dots\,800$\,MHz &$\omega^c$    & $100\,\dots\,800$\,MHz & range of cut off frequency \\
 \hline\xrowht[()]{15pt}
$v_k$    & $128\,\dots\,1024$    & $v_k$ &  $128\,\dots\,1024$ & division of the VCO's frequency\\
 \hline\xrowht[()]{15pt}
$K_k$   & $2\pi(8.1\,\dots\,162)10^7$\,radHz/V    & $K_k=K_k^{\textrm{\tiny VCO}}\, G_k^{\textrm{\tiny PD}}G^{\textrm{\tiny VGA}}G_k^{\textrm{\tiny LF}}/\textrm{2v}$ &  ... & coupling strength\\
%$K_k$   & $2\pi\cdot(4.97\,\dots\,39.76)\cdot10^5$\,Hz/V    & $K_k=K_k^{\textrm{\tiny VCO}}\, G_k^{\textrm{\tiny PD}}G^{\textrm{\tiny VGA}}G_k^{\textrm{\tiny LF}}/\textrm{2v}$ &  ... & coupling strength\\
 \hline
\end{tabular}
\end{flushleft}

For distance=423m, where $K_k^\textrm{min}$, $\omega^c=800$\,MHz, $v=8$  
\begin{flushleft}
\begin{tabular}{ |p{1.5cm}||p{3.9cm}|p{5.9cm}|p{3.0cm}|p{3cm}|  }
 \hline
$G_L(0)$   & $-4.3050 \cdot10^6$\,Hz    & $G_L(0)=\frac{K_k\;h'(-\Omega(\tau-\tau^f)+\beta_{kl})}{2v}$ &  $-4.3050 \cdot10^6$\,Hz & steady-state loop gain\\
 \hline
$G_L(i\gamma)$   & $-4.3050 \cdot10^6 \frac{p(i\gamma)}{i\gamma}$    & $G_L(i\gamma)=\frac{K_k\;h'(\Omega(\tau-\tau^f)+\beta_{kl})}{2v}\frac{p(i\gamma)}{i\gamma}$ &  $-4.3050 \cdot10^6 \frac{p(i\gamma)}{i\gamma}$ & loop gain\\
 \hline
\end{tabular}

\end{flushleft}




For distance=212m, where $K_k^\textrm{max}$, $\omega^c=100MHz$ and $v=4$ 
\begin{flushleft}
\begin{tabular}{ |p{1.5cm}||p{3.9cm}|p{5.9cm}|p{3.0cm}|p{3cm}|  }
 \hline
$G_L(0)$   & $-9.24095\cdot10^8 Hz$    & $G_L(0)=\frac{K_k\;h'-(\Omega(\tau-\tau^f)+\beta_{kl})}{2v}$ &  $-9.24095\cdot10^8$ & steady-state loop gain\\
 \hline
$G_L(i\gamma)$   & $-9.24095\cdot10^8 \frac{p(i\gamma)}{i\gamma}$    & $G_L(i\gamma)=\frac{K_k\;h'(\Omega(\tau-\tau^f)+\beta_{kl})}{2v}\frac{p(i\gamma)}{i\gamma}$ &  $-9.24095\cdot10^8 \frac{p(i\gamma)}{i\gamma}$ & loop gain\\
 \hline
\end{tabular}

\end{flushleft}

\subsection*{Open questions, need for discussion}

\begin{itemize}
 \item divider open questions: high vs. low division, cannot have both since the resulting frequency ranges in which all components need to operate need to be designed for specific frequency regimes; arguments for high division factors -- easy to be exchanged, less damping in case of wireless transmission (range, power consumption), arguments for low division factors -- integrated antennas might be possible (packaging)
 \item $\tau^\textrm{cc, tune}$ denotes the tunable part of the cross-coupling delay realized either within every input individually (hence each unidirectional connection can be controlled independently), in the output of the VCO's cross-coupling path, or... 
 \item when asking how large the delay can be for a certain setup so that we can still synchronize robustly, we should consider to use the gain-margin measure and provide the maximum time-delays that would guarantee a phase-margin of $\{30^\circ,\,40^\circ,\,50^\circ,\,60^\circ,\,\dots\}$; is it feasible to plot phase and gain margin as a function of the delay or other parameters
\end{itemize}




%\end{landscape}
\end{document}

